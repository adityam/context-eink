%D \module
%D   [     file=t-rssfeed,
%D      version=2012.05.25,
%D        title=\CONTEXT\ User Module,
%D     subtitle=eInk Style File for RSS Feeds,
%D       author=Aditya Mahajan,
%D         date=\currentdate,
%D    copyright=Aditya Mahajan,
%D        email=adityam <at> ieee <dot> org,
%D      license=Simplified BSD License]

\startmodule rssfeed

\unprotect

\usemodule[t][eink]
          [
            \c!alternative=kindle,
            mainfont={Dejavu Serif},
            sansfont={Dejavu Sans},
            monofont={Dejavu Sans Mono},
            \c!size=10pt,
          ]

% Each post is a new chapter
\setuphead[\v!chapter,\v!title]
          [
            \c!alternative=\v!middle,
            \c!style=\ssbfb,
            \c!before={\blank[2*\v!big,\v!force]},
            \c!after={\blank[4*\v!big]},
          ]

\setuphead[\v!section,\v!subject]
          [
            \c!style=\ssbfa,
            \c!before={\blank[\v!big]},
            \c!after={\blank[\v!big]},
          ]

\setuphead[\v!subsection,\v!subsubject]
          [
            \c!style=\ssbf,
            \c!before={\blank[\v!medium]},
            \c!after={\blank[\v!medium]},
          ]

\setuphead[\v!subsubsection,\v!subsubsubject]
          [
            \c!alternative=\v!text,
            \c!style=\ssbf,
            \c!before={\blank},
            \c!after=,
          ]

% Setup for feed titlepage
\setvariables
  [rssfeed]
  [
    title=,
    description=,
    link=,
    set={\setups{rssfeed:titlepage}},
  ]

\startsetups rssfeed:titlepage
  \startinterlude
      \getvariable{rssfeed}{title}
      \blank[2*big]
      \getvariable{rssfeed}{description}
      \vfill
      {\getvariable{rssfeed}{link}}
  \stopinterlude
\stopsetups

% Arxiv sometimes uses an explicit \emph command:
\definehighlight[emph][style=\em]

% Bugfix for figures
\let\normalexternalfigure\externalfigure
\unexpanded\def\externalfigure
    {\dodoubleargument\rssfeed_externalfigure}

\def\rssfeed_externalfigure[#1][#2]%
    {\normalexternalfigure[#1][\c!method=pdf, \c!scale=2000, #2]}


\protect
\stopmodule
