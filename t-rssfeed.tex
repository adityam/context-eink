%D \module
%D   [     file=t-rssfeed,
%D      version=2012.05.25,
%D        title=\CONTEXT\ User Module,
%D     subtitle=eInk Style File for RSS Feeds,
%D       author=Aditya Mahajan,
%D         date=\currentdate,
%D    copyright=Aditya Mahajan,
%D        email=adityam <at> ieee <dot> org,
%D      license=Simplified BSD License]

\startmodule rssfeed

\unprotect

\usemodule[t][eink]
          [
            \c!alternative=kindle,
            mainfont={Linux Libertine O},
            sansfont={Dejavu Sans},
            monofont={Inconsolata},
            \c!size=10pt,
          ]

\setupheadertexts[part][\setups{current:chapter}]

% Mostly, the typing environment is for code.
% Kindle has a small linewidth, so we use a smaller
% font for typing
\setuptyping
  [
    \c!style={\switchtobodyfont[8pt]},
    \c!before={\starttyping_background},
    \c!after={\stoptyping_background},
    \c!lines=\v!yes,
  ]

\definebackground[typing_background]
                 [
                   \c!background=\c!color,
                   \c!backgroundcolor=gray,
                 ]


% Each blog is a new part
\setuphead[\v!part]
          [
            \c!placehead=\v!yes,
            \c!incrementnumber=\v!list,
            \c!number=\v!no,
            \c!alternative=\v!middle,
            \c!before={\blank[force,3*big]},
          ]

\def\rssfeed_part#1#2%
    {
    \framed[\c!frame=\v!off, \c!align=\v!middle, \c!width=\v!broad, \c!foregroundstyle=\ssbfb]
      {#2
       \blank[2*big]
       \structureuservariable{description}
        \blank[4*big]
        \structureuservariable{link}}}

% Each post is a new chapter
\setuphead[\v!chapter,\v!title]
          [
            \c!alternative=\v!middle,
            \c!style=\ssbfb,
            \c!before={\blank[2*\v!big,\v!force]},
            \c!after={\blank[4*\v!big]},
            \c!number=\v!no,
          ]

\setuphead[\v!section,\v!subject]
          [
            \c!style=\ssbfa,
            \c!before={\blank[\v!big]},
            \c!after={\blank[\v!big]},
            \c!number=\v!no,
          ]

\setuphead[\v!subsection,\v!subsubject]
          [
            \c!style=\ssbf,
            \c!before={\blank[\v!medium]},
            \c!after={\blank[\v!medium]},
            \c!number=\v!no,
          ]

\setuphead[\v!subsubsection,\v!subsubsubject]
          [
            \c!alternative=\v!text,
            \c!style=\ssbf,
            \c!before={\blank},
            \c!after=,
            \c!number=\v!no,
          ]


% Bugfix for figures
\let\normalexternalfigure\externalfigure

% Weburls can be long. Create a md5 hash of the file, WITHOUT an extension.
\startluacode
  local cleaners = resolvers.schemes.cleaners
  function cleaners.md5auto(specification)
      return md5.hex(specification.original)
  end
\stopluacode

\enabledirectives[schemes.cleanmethod=md5auto]

% Web urls can have \type{%} signs in them.
\unexpanded\def\externalfigure
    {\startasciimode
     \dodoubleargument\rssfeed_externalfigure}

\def\rssfeed_externalfigure[#1][#2]%
    {\stopasciimode
        \normalexternalfigure[#1][\c!scale=2000, \c!method=auto, #2]}

% Feedburner includes a GIF image at the end of feeds. When using pandoc to
% convert HTML to ConTeXt, that gets translated to \placefigure[...]{...}{...} 
% and we get a box with "undefined" in it. To fix that, we set:

% \setupfloats[\c!width=\zeropoint, \c!height=\zeropoint]
% \startmessages  english  library: floatblocks
% 12: 
% \stopmessages

% Using asciimode is safer, but directly setting asciimode from
% a module does not work. But if we set asciimode, then math typsetting 
% does not work as most sites use $..$ for math. 
\appendtoks
  %\asciimode
\to\everystarttext

% Fonts
\definehighlight[emph][style=\em]
\definehighlight[strong][style=bold]


\protect
\stopmodule
