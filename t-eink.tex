%D \module
%D   [     file=t-eink,
%D      version=2012.04.07,
%D        title=\CONTEXT\ User Module,
%D     subtitle=eInk Style File,
%D       author=Aditya Mahajan,
%D         date=\currentdate,
%D    copyright=Aditya Mahajan,
%D        email=adityam <at> ieee <dot> org,
%D      license=Simplified BSD License]

\writestatus{loading}{eInk style (ver: 2012.04.07)}

\startmodule eink

\unprotect

\setupmodule
  [
    \c!alternative=kindle,
    mainfont={Tex Gyre Schola},
    sansfont={Tex Gyre Heros},
    monofont={Latin Modern Mono},
    mathfont={Xits},
    \c!size=,
  ]

% Load information about different devices
\usemodule[t][eink-devices]

% Global parameters
\edef\currenteinkdevice{\currentmoduleparameter\c!alternative}
\edef\p_eink_fontsize{\currentmoduleparameter\c!size}
\doifnothing{\p_eink_fontsize}
    {\edef\p_eink_fontsize{\einkdeviceparameter\c!size}}

% Define and set paper size based on the information stored
% in `t-eink-devices`.
\definepapersize
  [eink]
  [
    \c!height=\einkdeviceparameter\c!height,
    \c!width=\einkdeviceparameter\c!width,
  ]

\setuppapersize[eink]

% Set a layout with tiny margins and no headers and footers. Real-estate is
% at a premium on e-ink devices.

\setuplayout
  [
    \c!location=\v!middle,
    \c!topspace=5mm,
    \c!width=\v!middle,
    \c!height=\v!middle,
    \c!bottomspace=5mm,
    \c!bottomdistance=\zeropoint,
    \c!bottom=\zeropoint,
    \c!backspace=5mm,
    \c!cutspace=5mm,
    \c!leftmargin=1mm,
    \c!rightmargin=1mm,
    \c!header=\zeropoint,
    \c!footer=\zeropoint,
    \c!headerdistance=\zeropoint,
    \c!footerdistance=\zeropoint,
 ]


% Indent start of paragraphs
\setupindenting[\v!medium,\v!yes]

% But do no use extra space between paragraphs
% TODO: This can be made configurable based on the device
% size.
\setupwhitespace[\v!none]

% Unlike print documents with large margins, eink does no
% have any margins; so a overfull \hbox resulting from a bad
% word break will not be noticed. To avoid this, we set a high
% horizontal tolerance. This means that TeX does not work very hard
% to get good linebreaks. 
%
\setuptolerance[\v!tolerant, \v!stretch]

% By default, ConTeXt uses Math fonts for bullets in item lists.
% Use text font bullets to avoid loading the math font is the
% document does no use math.
\setupsymbolset[\v!text] 

% Try to avoid page-breaks just before the start of itemizations.
\setupitemize[\v!autointro,\v!intro]
\setupitemize[\c!margin=1em]

% Activate interaction, and instead of generating a 
% tablue of contents, just generate bookmaks. Most 
% e-ink readers will happily show bookmarks as table of contents.
\setupinteraction[\c!state=\v!start]
\placebookmarks[\v!chapter]

% Set document fonts. We deligate this task to the `simplefonts` module.
\usemodule  [simplefonts][\c!size=\p_eink_fontsize]

\setmainfont[\currentmoduleparameter{mainfont}][expansion=quality,protrusion=quality]
\setsansfont[\currentmoduleparameter{sansfont}][expansion=quality,protrusion=quality]
\setmonofont[\currentmoduleparameter{monofont}]
\setmathfont[\currentmoduleparameter{mathfont}]

% A easy to read style for chapter headings. By default, disable chapter
% numbering.
%
% TODO: Add settings for section and lower level headings as well.

\setuphead[\v!chapter]
          [
            \c!alternative=\v!middle,
            \c!style=\ssbfd,
            \c!number=\v!no,
            \c!before={\blank[2*\v!big,\v!force]},
            \c!after={\blank[4*\v!big]\placeinitial},
          ]

% Sometimes we need to typeset a title page, or a dedication page, or a part
% page. We use a generic `interlude` page for all such information, which
% typesets its content vertically and horizontally centered and using a large
% font.

\definestartstop
  [interlude]
  [
    before=\start_eink_interlude_makeup,
    after=\stop_eink_interlude_makeup,
  ]


\definemakeup
  [_eink_interlude_]
  [
    \c!top=\setups{interlude::before},
    \c!bottom=\setups{interlude::after},
    \c!color=,
    \c!align=\v!middle,
    \c!style={\ssbfc\setupinterlinespace},
  ]
    

\startsetups interlude::before
  \vfill \vfill
  \let\\=\crlf
\stopsetups

\startsetups interlude::after
  \vfill\vfill\vfill
\stopsetups

% Make sure that figures do not exceed the dimension of the page
\setupexternalfigures
  [
    \c!maxwidth=\textwidth,
    \c!maxheight=\textheight,
    \c!factor=\v!max,
  ]

\protect
\stopmodule
